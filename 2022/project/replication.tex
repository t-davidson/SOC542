% Options for packages loaded elsewhere
\PassOptionsToPackage{unicode}{hyperref}
\PassOptionsToPackage{hyphens}{url}
\PassOptionsToPackage{dvipsnames,svgnames,x11names}{xcolor}
%
\documentclass[
  12pt,
]{article}
\title{~STATISTICAL METHODS IN SOCIOLOGY II\\
\hspace*{0.333em}Final project instructions}
\author{Thomas Davidson}
\date{Spring 2022}

\usepackage{amsmath,amssymb}
\usepackage[]{mathpazo}
\usepackage{iftex}
\ifPDFTeX
  \usepackage[T1]{fontenc}
  \usepackage[utf8]{inputenc}
  \usepackage{textcomp} % provide euro and other symbols
\else % if luatex or xetex
  \usepackage{unicode-math}
  \defaultfontfeatures{Scale=MatchLowercase}
  \defaultfontfeatures[\rmfamily]{Ligatures=TeX,Scale=1}
\fi
% Use upquote if available, for straight quotes in verbatim environments
\IfFileExists{upquote.sty}{\usepackage{upquote}}{}
\IfFileExists{microtype.sty}{% use microtype if available
  \usepackage[]{microtype}
  \UseMicrotypeSet[protrusion]{basicmath} % disable protrusion for tt fonts
}{}
\makeatletter
\@ifundefined{KOMAClassName}{% if non-KOMA class
  \IfFileExists{parskip.sty}{%
    \usepackage{parskip}
  }{% else
    \setlength{\parindent}{0pt}
    \setlength{\parskip}{6pt plus 2pt minus 1pt}}
}{% if KOMA class
  \KOMAoptions{parskip=half}}
\makeatother
\usepackage{xcolor}
\IfFileExists{xurl.sty}{\usepackage{xurl}}{} % add URL line breaks if available
\IfFileExists{bookmark.sty}{\usepackage{bookmark}}{\usepackage{hyperref}}
\hypersetup{
  pdfauthor={Thomas Davidson},
  colorlinks=true,
  linkcolor={Maroon},
  filecolor={Maroon},
  citecolor={Blue},
  urlcolor={blue},
  pdfcreator={LaTeX via pandoc}}
\urlstyle{same} % disable monospaced font for URLs
\usepackage[margin=1in]{geometry}
\usepackage{graphicx}
\makeatletter
\def\maxwidth{\ifdim\Gin@nat@width>\linewidth\linewidth\else\Gin@nat@width\fi}
\def\maxheight{\ifdim\Gin@nat@height>\textheight\textheight\else\Gin@nat@height\fi}
\makeatother
% Scale images if necessary, so that they will not overflow the page
% margins by default, and it is still possible to overwrite the defaults
% using explicit options in \includegraphics[width, height, ...]{}
\setkeys{Gin}{width=\maxwidth,height=\maxheight,keepaspectratio}
% Set default figure placement to htbp
\makeatletter
\def\fps@figure{htbp}
\makeatother
\setlength{\emergencystretch}{3em} % prevent overfull lines
\providecommand{\tightlist}{%
  \setlength{\itemsep}{0pt}\setlength{\parskip}{0pt}}
\setcounter{secnumdepth}{-\maxdimen} % remove section numbering
\pagenumbering{arabic}
\usepackage{setspace}\onehalfspacing
\ifLuaTeX
  \usepackage{selnolig}  % disable illegal ligatures
\fi

\begin{document}
\maketitle

\hypertarget{replication-project}{%
\section{REPLICATION PROJECT}\label{replication-project}}

The goal of this project is to replicate the statistical analyses in a
published journal article. There are several important goals of this
analysis. First, it will be an opportunity to get hands-on experience
implementing statistical models in sociological research and will give
you a greater appreciation of the methodology used in published work
than you can get by simply reading a paper. Second, it will allow you to
question the original authors' decisions and explore how alternative
choices would have impacted the results. Third, depending on your choice
of paper, it could be an opportunity to critique or extend existing
work. Fourth, and finally, the replication task will highlight the
challenges involved in replication and reproducibility.

This document contains information on how to select a viable paper for
replication, the tasks required for the project, guidelines for writing
up and submitting the analysis, and a timeline for the remainder of the
semester.

\hypertarget{selecting-a-paper-to-replicate}{%
\section{SELECTING A PAPER TO
REPLICATE}\label{selecting-a-paper-to-replicate}}

I recommend selecting a paper related to your research interests.
Ideally, the paper should be published in a sociological journal (or
related social science field). The paper must include some form of
regression model. I strongly encourage you to find a paper with
replication code and data, although the availability will be highly
variable. If this is unavailable, you could also email the authors
(start with the corresponding author) and politely request the code
and/or data.
\href{https://orgtheory.wordpress.com/2015/08/11/sociologists-need-to-be-better-at-replication-a-guest-post-by-cristobal-young/}{Cristobal Young's 2015 class experiment}
found that 28\% of 53 authors contacted provided students with
replication packages.

The \href{https://dataverse.harvard.edu/}{Harvard Dataverse} and
\href{https://www.openicpsr.org/openicpsr/search/studies#}{OpenICPSR}
websites contain replication materials for many recent papers. You can
use the search bar on either website to find replication materials for
different journals. The replication materials often include some form of
README document that explains the structure of the replication data and
files. Note that some of the materials posted included data but not
replication code. You may want to contact the lead author to request the
code in such cases. In addition to these resources, I recommend
searching through relevant journals for articles of interest. Some
journals include downloadable replication packages on their websites and
some authors also host replication code on their personal websites and
Github repositories.

Avoid any papers that use restricted data as it is unlikely that you
will obtain the necessary data in a timely manner. You should also avoid
any papers that require running extensive supplementary code prior to
obtaining the replication dataset (e.g.~agent-based models) unless you
are confident you understand the approach and the relevant code. In some
cases, the replication materials might include Stata code
(e.g.~\texttt{do} files) or files written in another programming
language. Depending on the complexity of the code, it might be viable to
translate this into R. If you are comfortable doing so, you can run code
in other languages, but the final analyses (e.g.~regression models and
any output) should be reported using R in the RMarkdown document (see
below for further details).

As a general rule of thumb, you are more likely to find replication data
and code for more recently published papers.

\hypertarget{tasks}{%
\section{TASKS}\label{tasks}}

\begin{enumerate}
\def\labelenumi{\arabic{enumi}.}
\tightlist
\item
  \emph{Replicate a key finding of the paper.} Often multiple analyses
  are reported in quantitative papers. You do not need to reproduce
  every result in the paper. At a minimum, you should choose at least
  one regression model to reproduce.
\item
  \emph{Estimate a Bayesian version of the original model.} Use
  \texttt{stan\_glm} to estimate a Bayesian version of the model. You
  may use additional code to plot the results and show how Bayesian
  methods can be used to assess the result in additional ways (e.g.~Plot
  posterior distribution, posterior predictive checks, LOO-CV).
\item
  \emph{Examine robustness to alternative specifications.} The next step
  is to assess the robustness of the published result to alternative
  specifications. Your choices of alternative specifications should be
  motivated by your domain knowledge and statistical expertise. At a
  minimum, estimate three additional models, one for each of the
  following:

  \begin{itemize}
  \item
    I. A model with alternative variables (e.g.~add or remove controls,
    transformations)
  \item
    \begin{enumerate}
    \def\labelenumii{\Roman{enumii}.}
    \setcounter{enumii}{1}
    \tightlist
    \item
      A model with a different subset of the data (e.g.~removing
      outliers)
    \end{enumerate}
  \item
    \begin{enumerate}
    \def\labelenumii{\Roman{enumii}.}
    \setcounter{enumii}{2}
    \tightlist
    \item
      A model using a different estimator (e.g.~Poisson instead of OLS,
      Probit instead of logit)
    \end{enumerate}
  \end{itemize}
\item
  \emph{Write up the results.} Discuss any challenges related to the
  replication and whether you were able to reproduce the published
  results. For parts 2 and 3, discuss the results of the new analyses
  and if the alternative specifications result in any substantive
  changes.
\item
  \emph{(Optional) Extend the analyses of the original paper.} Is there
  a different kind of model you could estimate using the data that might
  provide insights into the research topic? Is there something different
  the author(s) could have done to get at the phenomenon under study? Is
  there a different question one could ask using the data?
\end{enumerate}

\hypertarget{paper-format}{%
\section{PAPER FORMAT}\label{paper-format}}

The replication analyses should be contained in an RMarkdown file. I
have provided a
\href{https://github.com/t-davidson/SOC542-S22/blob/main/project/template.Rmd}{template}
on the course website. This file will contain the code used to replicate
the analyses and any associated write up. You will submit the .Rmd file
and a rendered PDF. The PDF should include writing, tables, and figures.
Raw data and code chunks should \emph{not} be included in the final
output unless there is an important reason to show the code. References
should be provided in the text using the author-date format
(e.g.~(McElreath 2020)).

The paper should contain the following sections:

\begin{enumerate}
\def\labelenumi{\arabic{enumi}.}
\tightlist
\item
  \emph{Introduction}: Briefly discuss the paper you have chosen to
  replicate and the particular results you will be analysing.
\item
  \emph{Replication}: Present the initial replication and discuss your
  findings. This section should include a table or figure showing the
  replicated result (as close to the original paper as possible).
\item
  \emph{Bayesian replication}: Present the Bayesian replication of the
  model and discuss your findings. Pay close attention to any
  discrepancies between the Bayesian and frequentist models. Use tables
  and/or figures to communicate your results.
\item
  \emph{Alternative specifications}: Discuss each of the alternative
  specifications and use tables and/or figures to present your results.
  Use tables and/or figures to communicate your results.
\item
  \emph{Discussion}: Discuss your findings and reflect upon the
  replication exercise.
\item
  \emph{References}: Provide a section listing works cited.
\end{enumerate}

The final project including the .Rmd file, the PDF, and any associated
files and data should be added to a private Github repository. You will
submit the project by adding me (\texttt{t-davidson}) as a collaborator
to the project. I will then read the paper and try to replicate your
replication materials by knitting the RMarkdown file.

\hypertarget{timeline}{%
\section{TIMELINE}\label{timeline}}

\begin{itemize}
\tightlist
\item
  March 25: Select paper to replicate
\item
  May 2: In-class presentation
\item
  May 11: Replication paper due
\end{itemize}

\end{document}
