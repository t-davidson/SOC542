% Options for packages loaded elsewhere
\PassOptionsToPackage{unicode}{hyperref}
\PassOptionsToPackage{hyphens}{url}
\PassOptionsToPackage{dvipsnames,svgnames,x11names}{xcolor}
%
\documentclass[
  12pt,
]{article}
\title{Replication template}
\author{Author}
\date{May 11, 2022}

\usepackage{amsmath,amssymb}
\usepackage[]{mathpazo}
\usepackage{iftex}
\ifPDFTeX
  \usepackage[T1]{fontenc}
  \usepackage[utf8]{inputenc}
  \usepackage{textcomp} % provide euro and other symbols
\else % if luatex or xetex
  \usepackage{unicode-math}
  \defaultfontfeatures{Scale=MatchLowercase}
  \defaultfontfeatures[\rmfamily]{Ligatures=TeX,Scale=1}
\fi
% Use upquote if available, for straight quotes in verbatim environments
\IfFileExists{upquote.sty}{\usepackage{upquote}}{}
\IfFileExists{microtype.sty}{% use microtype if available
  \usepackage[]{microtype}
  \UseMicrotypeSet[protrusion]{basicmath} % disable protrusion for tt fonts
}{}
\makeatletter
\@ifundefined{KOMAClassName}{% if non-KOMA class
  \IfFileExists{parskip.sty}{%
    \usepackage{parskip}
  }{% else
    \setlength{\parindent}{0pt}
    \setlength{\parskip}{6pt plus 2pt minus 1pt}}
}{% if KOMA class
  \KOMAoptions{parskip=half}}
\makeatother
\usepackage{xcolor}
\IfFileExists{xurl.sty}{\usepackage{xurl}}{} % add URL line breaks if available
\IfFileExists{bookmark.sty}{\usepackage{bookmark}}{\usepackage{hyperref}}
\hypersetup{
  pdftitle={Replication template},
  pdfauthor={Author},
  colorlinks=true,
  linkcolor={Maroon},
  filecolor={Maroon},
  citecolor={Blue},
  urlcolor={black},
  pdfcreator={LaTeX via pandoc}}
\urlstyle{same} % disable monospaced font for URLs
\usepackage[margin=1in]{geometry}
\usepackage{graphicx}
\makeatletter
\def\maxwidth{\ifdim\Gin@nat@width>\linewidth\linewidth\else\Gin@nat@width\fi}
\def\maxheight{\ifdim\Gin@nat@height>\textheight\textheight\else\Gin@nat@height\fi}
\makeatother
% Scale images if necessary, so that they will not overflow the page
% margins by default, and it is still possible to overwrite the defaults
% using explicit options in \includegraphics[width, height, ...]{}
\setkeys{Gin}{width=\maxwidth,height=\maxheight,keepaspectratio}
% Set default figure placement to htbp
\makeatletter
\def\fps@figure{htbp}
\makeatother
\setlength{\emergencystretch}{3em} % prevent overfull lines
\providecommand{\tightlist}{%
  \setlength{\itemsep}{0pt}\setlength{\parskip}{0pt}}
\setcounter{secnumdepth}{-\maxdimen} % remove section numbering
\pagenumbering{arabic}
\usepackage{setspace}\doublespacing
\ifLuaTeX
  \usepackage{selnolig}  % disable illegal ligatures
\fi

\begin{document}
\maketitle

\hypertarget{introduction}{%
\section{INTRODUCTION}\label{introduction}}

\emph{Before adding to this document, click the Knit button above to
ensure that you are able to Knit it as a PDF. You should also modify the
author and title information above.}

This section can can be used to discuss the paper you have chosen to
replicate and the particular results you will be analyzing.

\hypertarget{replication}{%
\section{REPLICATION}\label{replication}}

Present the initial replication and discuss your findings. This section
should include a table or figure showing the replicated result (as close
to the original paper as possible).

Below is an example of a code chunk. Due to the defaults used above,
neither the code nor the output will get rendered to the final document.
You will need to modify the chunk options appropriately, depending on
your goal. These options are fine for loading and cleaning data, but
will need to be changed if you want to display a table or plot. I
recommend giving all of your chunks informative names.

You can also use LaTeX to render equations, either as part of a
sentencence like \(this\) or as a separate like like the following:

\[E = MC^2\]

\hypertarget{bayesian-replication}{%
\section{BAYESIAN REPLICATION}\label{bayesian-replication}}

Present the Bayesian extension of the model and discuss your findings.
Pay close attention to any discrepancies between the Bayesian and
frequentist models.

\hypertarget{alternative-specifications}{%
\section{ALTERNATIVE SPECIFICATIONS}\label{alternative-specifications}}

Discuss each of the alternative specifications and use tables and/or
figures to present your results. There should be at least three
alternatives (one changing data, one changing variables, one changing
model) but you are welcome to do more or to consider something more
systematic.

\hypertarget{discussion}{%
\section{DISCUSSION}\label{discussion}}

Discuss your findings and reflect upon the replication exercise.

\hypertarget{references}{%
\section{REFERENCES}\label{references}}

Provide a section listing works cited. You can add the references
directly in the text. It is also possible to include references directly
from a bibliography file, see
\url{https://bookdown.org/yihui/rmarkdown-cookbook/bibliography.html}

If you are not doing so already, I highly recommend using a reference
management program (I use Zotero as it is free and open source). It will
take a little time to set up but will save you a lot of time in the
long-run.

\end{document}
