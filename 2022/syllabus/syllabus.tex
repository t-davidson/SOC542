% Options for packages loaded elsewhere
\PassOptionsToPackage{unicode}{hyperref}
\PassOptionsToPackage{hyphens}{url}
%
\documentclass[
  10pt,
]{article}
\title{SOC542\\
\hspace*{0.333em}STATISTICAL METHODS IN SOCIOLOGY II\\
Rutgers University\\
\strut \\
\hspace*{0.333em}Syllabus}
\author{}
\date{\vspace{-2.5em}Spring 2022}

\usepackage{amsmath,amssymb}
\usepackage[]{mathpazo}
\usepackage{iftex}
\ifPDFTeX
  \usepackage[T1]{fontenc}
  \usepackage[utf8]{inputenc}
  \usepackage{textcomp} % provide euro and other symbols
\else % if luatex or xetex
  \usepackage{unicode-math}
  \defaultfontfeatures{Scale=MatchLowercase}
  \defaultfontfeatures[\rmfamily]{Ligatures=TeX,Scale=1}
\fi
% Use upquote if available, for straight quotes in verbatim environments
\IfFileExists{upquote.sty}{\usepackage{upquote}}{}
\IfFileExists{microtype.sty}{% use microtype if available
  \usepackage[]{microtype}
  \UseMicrotypeSet[protrusion]{basicmath} % disable protrusion for tt fonts
}{}
\makeatletter
\@ifundefined{KOMAClassName}{% if non-KOMA class
  \IfFileExists{parskip.sty}{%
    \usepackage{parskip}
  }{% else
    \setlength{\parindent}{0pt}
    \setlength{\parskip}{6pt plus 2pt minus 1pt}}
}{% if KOMA class
  \KOMAoptions{parskip=half}}
\makeatother
\usepackage{xcolor}
\IfFileExists{xurl.sty}{\usepackage{xurl}}{} % add URL line breaks if available
\IfFileExists{bookmark.sty}{\usepackage{bookmark}}{\usepackage{hyperref}}
\hypersetup{
  hidelinks,
  pdfcreator={LaTeX via pandoc}}
\urlstyle{same} % disable monospaced font for URLs
\usepackage[margin=1in]{geometry}
\usepackage{graphicx}
\makeatletter
\def\maxwidth{\ifdim\Gin@nat@width>\linewidth\linewidth\else\Gin@nat@width\fi}
\def\maxheight{\ifdim\Gin@nat@height>\textheight\textheight\else\Gin@nat@height\fi}
\makeatother
% Scale images if necessary, so that they will not overflow the page
% margins by default, and it is still possible to overwrite the defaults
% using explicit options in \includegraphics[width, height, ...]{}
\setkeys{Gin}{width=\maxwidth,height=\maxheight,keepaspectratio}
% Set default figure placement to htbp
\makeatletter
\def\fps@figure{htbp}
\makeatother
\setlength{\emergencystretch}{3em} % prevent overfull lines
\providecommand{\tightlist}{%
  \setlength{\itemsep}{0pt}\setlength{\parskip}{0pt}}
\setcounter{secnumdepth}{-\maxdimen} % remove section numbering
\linespread{1.05}
\ifLuaTeX
  \usepackage{selnolig}  % disable illegal ligatures
\fi

\begin{document}
\maketitle

\hypertarget{contact-and-office-hours}{%
\section{CONTACT AND OFFICE HOURS}\label{contact-and-office-hours}}

Instructor: Thomas Davidson

E-mail: \texttt{thomas.davidson@rutgers.edu}

Office hours: Wednesday 11:00-12:00pm, 109 Davison Hall, or by
appointment.

Teaching Assistant: Fred Traylor

E-mail: \texttt{frederic.traylor@rutgers.edu}

Office hours: Thursday 11:00-12:00pm, Davison Hall or Zoom.

\hypertarget{logistics}{%
\section{LOGISTICS}\label{logistics}}

Class meetings: Mondays 5:00-8:00 p.m. \emph{Our first meeting will be
on Zoom and we will return to in-person classes in Davison Hall from
1/31 onwards.}

Course website: \texttt{https://github.com/t-davidson/SOC542-S22}

\hypertarget{course-description}{%
\section{COURSE DESCRIPTION}\label{course-description}}

This is the second course in a two-semester sequence of graduate-level
statistics. The goal of the course is to provide an understanding of the
principles and application of statistics to sociological research. The
course begins with an overview of the quantitative approach to
sociological research and a review of basic statistics and ordinary
least squares regression. We then cover multiple regression, followed by
generalized linear models for binary, count, and categorical data.
Throughout the course, we will consider both frequentist and Bayesian
approaches to estimation and will explore various techniques for
improving the robustness and validity of statistical analyses. We will
pay close attention to the theoretical interpretations of statistical
models and emphasize effective and accurate scientific communication.

\hypertarget{prerequisites}{%
\section{PREREQUISITES}\label{prerequisites}}

Students should have taken SOC541 or an equivalent introduction to
statistics. The course assumes some basic familiarity with data
manipulation and visualization in R and RStudio.

\hypertarget{learning-goals}{%
\section{LEARNING GOALS}\label{learning-goals}}

By the end of the semester, students will:

\begin{itemize}
\tightlist
\item
  Be proficient in preparing datasets, running basic descriptive
  analyses, and producing informative data visualizations using R.
\item
  Understand the conceptual underpinnings and assumptions of multiple
  regression and generalized linear models
\item
  Be able to implement and interpret various different forms of
  regression models
\item
  Be familiar with and proficient in handling interaction effects,
  non-linear relationships, and violations of assumptions in multiple
  regression analyses
\item
  Understand and apply frequentist and Bayesian methods for estimation
\item
  Replicate, reassess, and extend quantitative research published in
  leading sociological journals
\end{itemize}

\hypertarget{assessment}{%
\section{ASSESSMENT}\label{assessment}}

\begin{enumerate}
\def\labelenumi{\arabic{enumi}.}
\item
  \emph{Homework assignments} (50\%): Homework assignments will be used
  to assess comprehension of materials covered in class. Assignments
  will be submitted using Github Classroom. Students can work together
  on the problem sets but must submit assignments individually.
\item
  \emph{Replication paper} (40\%). Each student will write a replication
  paper. The objectives of the replication will be to (a) reproduce a
  finding published in a leading sociological journal, (b) to assess the
  robustness of the reporting finding to alternative specifications, and
  (c) to extend the original analysis.
\item
  \emph{Class presentation} (10\%). Each student will present the
  findings of their replication paper to the class during one of the
  last two class sessions.
\end{enumerate}

\hypertarget{readings}{%
\section{READINGS}\label{readings}}

There are weekly reading assignments for this course. Students are
expected to complete the assigned readings \emph{before} class. Students
must purchase copies of the two required texts. Some weeks will also
include additional papers published in academic journals. The
recommended texts provide useful background material on data analysis,
visualization, and causal inference. All three are available for free
online (links are embedded in the book titles).

\emph{Required}

\begin{itemize}
\tightlist
\item
  Gelman, Andrew, Jennifer Hill, and Aki Vehtari. 2020. \emph{Regression
  and Other Stories}. Cambridge University Press.
\item
  McElreath, Richard. 2020. \emph{Statistical Rethinking: A Bayesian
  Course with Examples in R and Stan}. 2nd ed.~Chapman and Hall/CRC.
\end{itemize}

\emph{Recommended}

\begin{itemize}
\item
  Wickham, Hadley, and Garrett Grolemund. 2016.
  \textit{\href{https://r4ds.had.co.nz/}{R for Data Science: Import, Tidy, Transform, Visualize, and Model Data}}.
  (\emph{R4DS}). O'Reilly Media, Inc.
\item
  Healy, Kieran. 2018.
  \textit{\href{https://socviz.co/}{Data Visualization: A Practical Introduction}}.
  Princeton University Press.
\item
  Cunningham, Scott. 2021.
  \textit{\href{https://mixtape.scunning.com/}{Causal Inference: The Mixtape}}.
  Yale University Press.
\end{itemize}

\hypertarget{policies}{%
\section{POLICIES}\label{policies}}

The Rutgers Sociology Department strives to create an environment that
supports and affirms diversity in all manifestations, including race,
ethnicity, gender, sexual orientation, religion, age, social class,
disability status, region/country of origin, and political orientation.
We also celebrate diversity of theoretical and methodological
perspectives among our faculty and students and seek to create an
atmosphere of respect and mutual dialogue. We have zero tolerance for
violations of these principles and have instituted clear and respectful
procedures for responding to such grievances.

Students must abide by the Code of Student Conduct and the university's
Academic Integrity Policy at all times, including during lectures and in
participation online. Violations of academic integrity will result in
disciplinary action.

In accordance with University policy, if you have a documented
disability and require accommodations to obtain equal access in this
course, please contact me during the first week of classes. Students
with disabilities must be registered with the Office of Student
Disability Services and must provide verification of their eligibility
for such accommodations.

I will also make additional accommodations due to the COVID-19 pandemic.
If you or your family are affected in any way that impedes your ability
to participate in this course, please contact me as soon as you can so
that we can make necessary arrangements.

\hypertarget{outline}{%
\section{OUTLINE}\label{outline}}

\hypertarget{week-1---january-24---statistics-review-and-course-overview}{%
\subsection{Week 1 - January 24 - Statistics review and course
overview}\label{week-1---january-24---statistics-review-and-course-overview}}

\emph{Readings}

\begin{itemize}
\tightlist
\item
  Gelman, Hill, and Vehtari (GHV) Chapters 1-5
\item
  McElreath Chapter 1
\item
  Imbens, Guido W. 2021. ``Statistical Significance, p-Values, and the
  Reporting of Uncertainty.'' \emph{Journal of Economic Perspectives} 35
  (3): 157--74. \url{https://doi.org/10.1257/jep.35.3.157}.
\end{itemize}

\emph{Recommended}

\begin{itemize}
\tightlist
\item
  Raftery, Adrian E. 2000. ``Statistics in Sociology, 1950--2000.''
  \emph{Journal of the American Statistical Association} 95 (450):
  654--61. \url{https://doi.org/10.1080/01621459.2000.10474245}.
\item
  Cunningham p.~16-36
\end{itemize}

\hypertarget{week-2---january-30---linear-regression-with-a-single-predictor}{%
\subsection{Week 2 - January 30 - Linear regression with a single
predictor}\label{week-2---january-30---linear-regression-with-a-single-predictor}}

\emph{Readings}

\begin{itemize}
\tightlist
\item
  GHV 6-7
\item
  Freese, Jeremy, and David Peterson. 2017. ``Replication in Social
  Science.'' \emph{Annual Review of Sociology} 43: 147--65.
  \url{https://doi.org/10.1146/annurev-soc-060116-053450}
\end{itemize}

\emph{Recommended}

\begin{itemize}
\tightlist
\item
  Cunningham p.~37-76
\end{itemize}

\emph{Homework 1 released, due 2/11}

\hypertarget{week-3---february-7---frequentist-and-bayesian-approaches-to-estimation}{%
\subsection{Week 3 - February 7 - Frequentist and Bayesian approaches to
estimation}\label{week-3---february-7---frequentist-and-bayesian-approaches-to-estimation}}

\emph{Readings}

\begin{itemize}
\tightlist
\item
  McElreath 2-3
\item
  GHV 8-9
\end{itemize}

\emph{Recommended}

\begin{itemize}
\tightlist
\item
  Gelman, Andrew. 2014. ``How Bayesian Analysis Cracked the Red-State,
  Blue-State Problem.'' \emph{Statistical Science} 29 (1).
  \url{https://doi.org/10.1214/13-STS458}.
\item
  Kruschke, John K., and Torrin M. Liddell. 2018. ``The Bayesian New
  Statistics: Hypothesis Testing, Estimation, Meta-Analysis, and Power
  Analysis from a Bayesian Perspective.'' \emph{Psychonomic Bulletin \&
  Review} 25 (1): 178--206.
  \url{https://doi.org/10.3758/s13423-016-1221-4}.
\item
  Lynch, Scott M., and Bryce Bartlett. 2019. ``Bayesian Statistics in
  Sociology: Past, Present, and Future.'' \emph{Annual Review of
  Sociology} 45 (1): 47--68.
  \url{https://doi.org/10.1146/annurev-soc-073018-022457}.
\item
  Lundberg, Ian, Rebecca Johnson, and Brandon M Stewart. 2021. ``What Is
  Your Estimand? Defining the Target Quantity Connects Statistical
  Evidence to Theory.'' \emph{American Sociological Review} 86 (3):
  532--65. \url{https://doi.org/10.1177/00031224211004187}.
\end{itemize}

\hypertarget{week-4---february-14---multiple-regression}{%
\subsection{Week 4 - February 14 - Multiple
regression}\label{week-4---february-14---multiple-regression}}

\emph{Readings}

\begin{itemize}
\tightlist
\item
  GHV 10.1-10.2, 10.7-11.6
\item
  McElreath 4-4.4, 5-5.2
\end{itemize}

\emph{Homework 2 released, due 2/27}

\hypertarget{week-5---february-21---dummy-categorical-and-non-linear-variables}{%
\subsection{Week 5 - February 21 - Dummy, categorical, and non-linear
variables}\label{week-5---february-21---dummy-categorical-and-non-linear-variables}}

\emph{Readings}

\begin{itemize}
\tightlist
\item
  GHV 10.3-10.6, 12-12.5, 12.7-12.8
\item
  McElreath 4.5.1, 5.3-5.4
\end{itemize}

\emph{Recommended}

\begin{itemize}
\tightlist
\item
  Johfre, Sasha Shen, and Jeremy Freese. 2021. ``Reconsidering the
  Reference Category.'' \emph{Sociological Methodology} 51 (2): 253--69.
  \url{https://doi.org/10.1177/0081175020982632}.
\end{itemize}

\hypertarget{week-6---february-28---interactions}{%
\subsection{Week 6 - February 28 -
Interactions}\label{week-6---february-28---interactions}}

\emph{Readings}

\begin{itemize}
\tightlist
\item
  GHV 10.3, 12.2
\item
  McElreath 8
\end{itemize}

\emph{Recommended}

\begin{itemize}
\tightlist
\item
  Mize, Trenton. 2019. ``Best Practices for Estimating, Interpreting,
  and Presenting Nonlinear Interaction Effects.'' \emph{Sociological
  Science} 6: 81--117. \url{https://doi.org/10.15195/v6.a4}.
\end{itemize}

\hypertarget{week-7---march-7---model-checking-and-missing-data}{%
\subsection{Week 7 - March 7 - Model checking and missing
data}\label{week-7---march-7---model-checking-and-missing-data}}

\emph{Readings}

\begin{itemize}
\tightlist
\item
  GHV 11.7-11.9, 17.3-17.8
\item
  McElreath 7
\end{itemize}

\emph{Recommended}

\begin{itemize}
\tightlist
\item
  Young, Cristobal, and Katherine Holsteen. 2017. ``Model Uncertainty
  and Robustness: A Computational Framework for Multimodel Analysis.''
  \emph{Sociological Methods \& Research} 46 (1): 3--40.
  \url{https://doi.org/10.1177/0049124115610347}.
\item
  Slez, Adam. 2017. ``The Difference Between Instability and
  Uncertainty: Comment on Young and Holsteen (2017).''
  \emph{Sociological Methods \& Research} 48 (2): 400--430.
  \url{https://doi.org/10.1177/0049124117729704}.
\item
  Muñoz, John, and Cristobal Young. 2018. ``We Ran 9 Billion
  Regressions: Eliminating False Positives through Computational Model
  Robustness.'' \emph{Sociological Methodology} 48 (1): 1--33.
  \url{https://doi.org/10.1177/0081175018777988}.
\item
  Western, Bruce. 2018. ``Comment: Bayes, Model Uncertainty, and
  Learning From Data.'' \emph{Sociological Methodology}
  \url{https://doi.org/10.1177/0081175018799095}.
\item
  Molina, Mario, and Filiz Garip. 2019. ``Machine Learning for
  Sociology.'' \emph{Annual Review of Sociology} 45: 27--45.
  \url{https://doi.org/10.1146/annurev-soc-073117-041106}.
\item
  McElreath 15.2
\end{itemize}

\hypertarget{spring-break---no-class}{%
\subsection{\texorpdfstring{\emph{SPRING BREAK - No
class}}{SPRING BREAK - No class}}\label{spring-break---no-class}}

\hypertarget{week-8---march-21---glms-i-binary-outcomes-and-logistic-regression}{%
\subsection{Week 8 - March 21 - GLMs I: Binary outcomes and logistic
regression}\label{week-8---march-21---glms-i-binary-outcomes-and-logistic-regression}}

\emph{Readings}

\begin{itemize}
\tightlist
\item
  GHV 13, 15.1, 15.4
\item
  McElreath 10.2-10.4, 11.1
\end{itemize}

\emph{Recommended}

\begin{itemize}
\tightlist
\item
  McElreath 10.1
\item
  Battey, H. S., D. R. Cox, and M. V. Jackson. 2019. ``On the Linear in
  Probability Model for Binary Data.'' \emph{Royal Society Open Science}
  6 (5): 190067. \url{https://doi.org/10.1098/rsos.190067}.
\end{itemize}

\emph{Homework 4 released, due 4/1}

\hypertarget{week-9---march-28---glms-ii-logistic-regression-and-marginal-effects}{%
\subsection{Week 9 - March 28 - GLMs II: Logistic regression and
marginal
effects}\label{week-9---march-28---glms-ii-logistic-regression-and-marginal-effects}}

\emph{Readings}

\begin{itemize}
\tightlist
\item
  GHV 14
\end{itemize}

\emph{Recommended} - Long, J. Scott, and Sarah A. Mustillo. 2018.
``Using Predictions and Marginal Effects to Compare Groups in Regression
Models for Binary Outcomes.'' \emph{Sociological Methods \& Research} 50
(3): 1284--1320. \url{https://doi.org/10.1177/0049124118799374}. - Mize,
Trenton. 2019. ``Best Practices for Estimating, Interpreting, and
Presenting Nonlinear Interaction Effects.'' \emph{Sociological Science}
6: 81--117. \url{https://doi.org/10.15195/v6.a4}.

\hypertarget{week-10---april-4---glms-iii-count-outcomes-and-overdispersion}{%
\subsection{Week 10 - April 4 - GLMs III: Count outcomes and
overdispersion}\label{week-10---april-4---glms-iii-count-outcomes-and-overdispersion}}

\emph{Readings}

\begin{itemize}
\tightlist
\item
  GHV 15.2-15.3, 15.8
\item
  McElreath 11.2, 12.1-12.2
\end{itemize}

\hypertarget{week-11---april-11---glms-iv-categorical-and-ordered-outcomes}{%
\subsection{Week 11 - April 11 - GLMs IV: Categorical and ordered
outcomes}\label{week-11---april-11---glms-iv-categorical-and-ordered-outcomes}}

\emph{Readings}

\begin{itemize}
\tightlist
\item
  GHV 15.5
\item
  McElreath 11.3, 12.3-12.5
\end{itemize}

\emph{Homework 5 released, due 4/22}

\hypertarget{week-12---april-18---clustered-data}{%
\subsection{Week 12 - April 18 - Clustered
data}\label{week-12---april-18---clustered-data}}

\emph{Readings}

\begin{itemize}
\tightlist
\item
  McElreath 13-13.3, 13.5-13.6, 14-14.2
\item
  Bell, Andrew, Malcolm Fairbrother, and Kelvyn Jones. 2019. ``Fixed and
  Random Effects Models: Making an Informed Choice.'' \emph{Quality \&
  Quantity} 53 (2): 1051--74.
  \url{https://doi.org/10.1007/s11135-018-0802-x}.
\item
  King, Gary, and Margaret E. Roberts. 2015. ``How Robust Standard
  Errors Expose Methodological Problems They Do Not Fix, and What to Do
  About It.'' \emph{Political Analysis} 23 (02): 159--79.
  \url{https://doi.org/10.1093/pan/mpu015}.
\end{itemize}

\emph{Recommended}

\begin{itemize}
\tightlist
\item
  Keele, Luke, and Nathan J. Kelly. 2006. ``Dynamic Models for Dynamic
  Theories: The Ins and Outs of Lagged Dependent Variables.''
  \emph{Political Analysis} 14 (02): 186--205.
  \url{https://doi.org/10.1093/pan/mpj006}.
\item
  De Boef, Suzanna, and Luke Keele. 2008. ``Taking Time Seriously.''
  \emph{American Journal of Political Science} 52 (1): 184--200.
  \url{https://doi.org/10.1111/j.1540-5907.2007.00307.x}
\item
  Rüttenauer, Tobias. 2019. ``Spatial Regression Models: A Systematic
  Comparison of Different Model Specifications Using Monte Carlo
  Experiments.'' \emph{Sociological Methods \& Research}, November,
  004912411988246. \url{https://doi.org/10.1177/0049124119882467}.
\item
  Chi, Guangqing, and Jun Zhu. 2008. ``Spatial Regression Models for
  Demographic Analysis.'' \emph{Population Research and Policy Review}
  27 (1): 17--42. \url{https://doi.org/10.1007/s11113-007-9051-8}.
\item
  Dow, Malcolm M., Michael L. Burton, and Douglas R. White. 1982.
  ``Network Autocorrelation: A Simulation Study of a Foundational
  Problem in Regression and Survey Research.'' \emph{Social Networks} 4
  (2): 169--200. \url{https://doi.org/10.1016/0378-8733(82)90031-4}.
\item
  DellaPosta, Daniel, Yongren Shi, and Michael Macy. 2015. ``Why Do
  Liberals Drink Lattes?'' \emph{American Journal of Sociology} 120 (5):
  1473--1511. \url{https://doi.org/10.1086/681254}.
\end{itemize}

\hypertarget{week-13---april-25---causal-inference-using-observational-data}{%
\subsection{Week 13 - April 25 - Causal inference using observational
data}\label{week-13---april-25---causal-inference-using-observational-data}}

\emph{Readings}

\begin{itemize}
\tightlist
\item
  GHV 18-21
\item
  Cunningham p.~96-198 (Chapters 3-5),
\end{itemize}

\emph{Recommended}

\begin{itemize}
\tightlist
\item
  Cunningham \emph{skim} p.~241-509 (Chapters 6-9)
\item
  McElreath 6
\item
  King, Gary, and Richard Nielsen. 2019. ``Why Propensity Scores Should
  Not Be Used for Matching.'' \emph{Political Analysis} 27 (4): 435--54.
  \url{https://doi.org/10.1017/pan.2019.11}.
\end{itemize}

\hypertarget{week-14---may-2---student-presentations}{%
\subsection{Week 14 - May 2 - Student
presentations}\label{week-14---may-2---student-presentations}}

\end{document}
